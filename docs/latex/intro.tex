\section{Introduction}
Sequence labelling is broad category of tasks which focus on assigning labels
to sequential data formats. In the field of natural language processing (NLP),
sequence data types are usually sentences and the problem boils down to
labelling different parts of sentences such as whole sentences, words or
characters.  Part-of-speech (POS) tagging and named entity recognition (NER)
are two widely research problems in the NLP domain. The former one has the goal
of predicting POS for every word in a sentence and the latter one tries to 
predict the pre-defined entity class.

Traditional state-of-the-art models are linear statistical models, including
Hidden Markov Models (HMM)\cite{hmm} and Conditional Random Fields
(CRF)\cite{crf}. All of the former models rely mostly on hand-crafted features
and task-specific resources. Some of the examples include designing word
spelling features for POS tagging problems and creating orthographic features
and external resources such as gazetteers in NER. One obvious problem with
these approaches is the requirement of the task-specific resources which are not
always available and can be hard to obtain.

With the advent of more powerful GPU resources at the beginning of the decade,
sequence labelling problems have become more oriented towards Deep Neural
Networks (DNNs) models. Recurrent Neural Networks (RNN)\cite{rnn} and their
improved variants, Long-Short Term Memory (LSTM)\cite{lstm} and Gated Recurrent
Units (GRU)\cite{gru}, have outperformed many traditional approaches and
continue to show the most promising results in sequence labelling types of
problems.

End-to-end Sequence Labeling via Bi-directional LSTM-CNNs-CRF\cite{impl} is a
paper where the authors propose a novel approach for sequence labelling tasks.
They propose a  end-to-end neural network model which requires no task-specific
resources, feature engineering, or data pre-processing beyond pre-trained word
embeddings. They state that their approach can be used on a wide range of
sequence labelling tasks on different languages and domains. Described model
consists of three main components: Convolutional Neural Network
(CNN)\cite{cnn} to encode character-level features, bi-directional
LSTM\cite{bi-lstm}, and a CRF layer at the end which jointly decodes the labels
for the whole sentence. The original paper was evaluated on two widely known
datasets: Penn Treebank WSJ\cite{wsj} and NER on English data
from the CoNLL 2003 shared task\cite{ner}. My experiments used slightly
modified version of the treebank dataset. As the WSJ Treebank is not free for
download, I decided to use the sample provided with the NLTK Python
package\cite{nltk}. The sample consists of one section from the original
treebank which accounts to around 10$\%$of the complete dataset.
Authors state that their system achieves 97.55$\%$ accuracy for the POS tagging
and 91.21$\%$ F1 score for NER, both on the test sets. The objective of the seminar
was to replicate the results from the original paper by first recreating the
model architecture in the Google Tensorflow library\cite{tens}, followed by repeating
the experiments using the above described datasets.


